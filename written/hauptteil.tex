\chapter{Grundlagen}
\label{chap:hess-smith}


\section{Hess-Smith-Panelverfahren}
\begin{figure}[!h]
\begin{center} \includegraphics[scale=0.3]{figures/zaehlrichtung.png} \end{center}
\caption{Zur Definition der Zählrichtung der Panels sowie der Anströmgeschwindigkeit und deren Winkel}
\label{fig:zaehlrichtung}
\end{figure}
Das Hess-Smith-Panelverfahren ist ein Verfahren zur Berechnung von ebenen Potentialströmungen um geschlossene Körper mit Auftrieb. Ein kontinuierlicher zweidimensionaler Körper mit der Umfangkurve $\mathcal{C}$ wird dabei zunächst durch  $n + 1$ diskrete Datenpunkte $\vec x_i = (x_i, y_i)$ dargestellt. Die Zählrichtung dieser Punkte sei als im Uhrzeigersinn festgelegt (siehe Abb. \ref{fig:zaehlrichtung}). Für das Profil gilt somit die Periodizität $\vec x_n = \vec x_0$.\\
Die Kontur des Profils wird nun durch eine endliche Anzahl Verbindungsgeraden zwischen zwei benachbarten Punkten $\vec x_i, \vec x_{i+1}$, den Panelen $\mathcal{C}_i$ (Abb. \ref{fig:panel}) angenähert. Dabei wird jedes Panel durch die folgenden Parameter charakterisiert:
\begin{itemize}
\item Den Mittelpunkten auf beiden Achsen: 
\begin{equation}
X_i =  \frac{x_{n-i}+x_{n-i-1}}{2}, \; Y_i =  \frac{y_{n-i}+y_{n-i-1}}{2},
\end{equation}
\item seinem Neigungswinkel: 
\begin{equation}
\theta_i =  \left( \frac{y_{n-i-1} - y_{n-i}}{x_{n-i-1} - x_{n-i}} \right),
\end{equation}
\item und seiner Länge: 
\begin{equation}
l_i =  \sqrt{(x_{n-i-1} - x_{n-i})^2 + (y_{n-i-1} - y_{n-i})^2}.
\end{equation}
\end{itemize}


\begin{figure}
\begin{center} \includegraphics[scale=0.5]{figures/panel.png} \end{center}
\caption{Zur Definition eines Panels}
\label{fig:panel}
\end{figure}

Im Hess-Smith-Panelverfahren sei nun jedes dieser Panels mit einer vorerst unbestimmten, konstanten Quelldichte $q_i$ behaftet, welche an einem Punkt $\vec r_i =  \vec x - \hat x_i(s)$ zum Geschwindigkeitspotential den Beitrag
\begin{equation}
\label{eqn:potq}
\phi_i^{(q)} =  \frac{q_i}{2 \pi } \int_{\mathcal{C_i}} ds \ln r_i
\end{equation}
liefert. $s$ ist dabei die Koordinate der Bogenlänge, welche das gesamte Profil parametrisiert. \\
Ebenso liefert die Profilanströmung mit der konstanten Geschwindigkeit $V_{\infty}$ unter dem Winkel $\alpha $ ein Potential
\begin{equation}
\label{eqn:potv}
\phi_{\infty} =  V_{\infty} (\cos \alpha x + \sin \alpha y).
\end{equation}
Aus der Summe über alle Panelbeiträge aus (\ref{eqn:potq}) und (\ref{eqn:potv}) ergibt sich das Gesamtgeschwindigkeitspotential zu
\begin{equation}
\label{eqn:potnovortex}
\phi(\vec x) =  V_{\infty} (\cos \alpha x + \sin \alpha y) + \sum_{i=0}^{n-1} \phi_i^{(q)}.
\end{equation}
Daraus folgt für die Geschwindigkeit im Punkte $\vec x$ 
\begin{equation}
\vec v ( \vec x) =  \nabla  \phi (\vec x).
\end{equation}
Auf der Oberfläche des Profils muss die Strömung die Gleitbedingung erfüllen; damit lassen sich $n$ Gleichungen aufstellen, um diese Bedingung zu modellieren. Wir verlangen, dass die Normalkomponente des Geschwindigkeitsvektors,
\begin{equation}
v_j^{(n)} =  \vec v(\vec X_j) \cdot \vec n_j,
\end{equation}
verschwindet. Dabei ist $\vec n_j$ der Normalvektor auf das Panel $\mathcal{C}_j$.\\
Wir erhalten ein System von $n$ Gleichungen
\begin{equation}
\sum_{j=0}^{n-1} M_{ij}^{(q)}g_j =  b_i,
\end{equation}
mit
\begin{equation}
M_{ij}^{(q)} = \frac{1}{q_j} \nabla \phi_j^{(q)} (\vec X) \cdot \vec n_i,
\end{equation}
\begin{equation}
b_i =  -V_{\infty} \left( \begin{matrix} \cos \alpha \\ \sin \alpha \end{matrix} \right) \cdot \vec n_i,
\end{equation}
aus dessen Lösung wir die Quellstärken $q_i$ für jedes Panel ermitteln können. \\
Mithilfe der Tangentialkomponenten des Geschwindigkeitsvektors
\begin{equation}
\label{eqn:vt}
v_j^{(t)} =  \vec v(X_j) \cdot \frac{\vec x_{j+1}-\vec x_j}{|\vec x_{j+1}-\vec x_j|},
\end{equation}
können wir nun den Druckbeiwert
\begin{equation}
c_{p_j} =  1 - \left( \frac{v_j^{(t)}}{V_{\infty}}\right)
\end{equation}
zum Panel $\mathcal{C}_j$ bestimmen.
\subsection{Berücksichtigung einer Wirbelbewegung}
Wollen wir den Auftrieb berücksichtigen, so reicht die reine Quellbelegung des Profils noch nicht aus. \\
Im Hess-Smith Verfahren nehmen wir für das gesamte Profil eine einzige konstante Wirbelbelegung $\gamma$ für alle Panels an. Damit wird das Potential (\ref{eqn:potnovortex}) erweitert um Beiträge
\begin{equation}
\phi_i^{(w)} =  \frac{\gamma}{2 \pi } \int ds \theta_i,
\end{equation}
zu
\begin{align}
\phi(\vec x) &=  V_{\infty} (\cos \alpha x + \sin \alpha y) + \sum_{i=0}^{n-1} \left( \phi_i^{(q)} + \phi_i^{(w)} \right) \nonumber \\
&= V_{\infty} (\cos \alpha x + \sin \alpha y) + \sum_{i=0}^{n-1} \int_{\mathcal{C}_i} \left( \frac{q_i}{2\pi } \ln r_i - \frac{\gamma}{2\pi } \theta_{i} \right) ds ´´
\end{align}
Die Gleitbedingung bleibt erhalten, allerdings muss das Gleichungssystem aber entsprechend dem hinzugekommenen Geschwindigkeitsbeitrag erweitert werden. Für die neue Unbekannte $\gamma$ fehlt also eine Gleichung. Diese ergibt sich aus der Kutta-Bedingung, welche besagt, dass es an der Hinterkante des Profils keine Umströmung gibt.\\
Zur Modellierung der Kutta-Bedingung wählen wir
\begin{equation}
v_0^{(t)} =  -v_n^{(t)}.
\end{equation} 
Daraus folgt, dass die Tangentialkomponenten des Geschwindigkeitsvektors in den Mittelpunkten der Panels, welche die Profilhinterkante bilden, vom Betrag gleich groß und gegenläufig gerichtet sein muss (siehe Abb. \ref{fig:kutta}). \\
Nach Lösung des erhaltenen linearen Gleichungssystems mit $q_n = \gamma$ können wir nun den Auftriebsbeiwert $c_a$ ermitteln. Dieser ergibt sich mit $t$, der Proftiltiefe und $l_i$, der Länge des Panels $\mathcal{C}_i$, approximiert in folgender Form:
\begin{equation}
c_a \approx \frac{2}{V_{\infty t}}\sum_{i=0}^{n-1} v_i^{(t)} l_i.
\end{equation}
\begin{figure}
\begin{center} \includegraphics[scale=0.7]{figures/kutta.png} \end{center}
\caption{Zur Kutta-Bedingung am Beispiel eines NACA0012-Profils - am Punkt der Hinterkante sind die Tangentialgeschwindigkeiten genau entgegengerichtet}
\label{fig:kutta}
\end{figure}
\cite{Hess:1966} \cite{Cebeci:1999} \cite{Alonso:2005}

\newpage
\chapter{Bericht der Arbeit}
Im Folgenden werden die wesentlichen Erkenntnisse der Arbeit präsentiert.
\section{Vorbereitung der Daten}
Die Daten der Profile wurden der UIUC Airfoil Data Site\footnote{\url{https://m-selig.ae.illinois.edu/ads/coord_database.html}} entnommen. Die Daten lagen dabei überwiegend im Selig- oder Lednicer-Format vor.\footnote{Eine Gegenüberstellung der beiden Formate ist in \nameref{appendix:a} zu finden.}
\\
Im Lednicer-Format wird in der ersten Zeile die Bezeichnung des Profils angegeben. In der zweiten Zeile wird die Anzahl der Koordinatenpaare der Ober- und Unterseite angegeben. Ab der dritten Zeile werden die Koordinaten der Panelenden $(x_i,y_i)$ von $x=0$ bis $x=1$ für die Oberseite des Profils angegeben. Danach folgt eine Leerzeile. Danach werden die Koordinaten der Panelenden $(x_i,y_i)$ von $x=0$ bis $x=1$ für die Unterseite des Profils angegeben.
\\
Im Selig-Format wird in der ersten Zeile die Bezeichnung des Profils angegeben. Ab der zweiten Zeile werden die Koordinaten der Panelenden $(x_i,y_i)$ im Gegenuhrzeigersinn, beginnend bei $x=1$ angegeben.
\\\\
Es wurde eine Routine geschrieben, welche Daten im Lednicer-Format in das Selig-Format überführt. Dies wurde bewerkstelligt, indem die Koordinatenpaare der Oberseite invertiert wurden. Anschließend wurden aus den Daten die Header-Zeilen entfernt. 
\section{Untersuchungen am  Kreiszylinder}
\subsection{Achtseitiger Kreiszylinder}
\subsubsection{Berechnung der Systemmatrix}
\subsubsection{Lösung des wirbellosen Gleichungssystems}
\subsection{Untersuchungen an variierenden Kreiszylindern}
\subsubsection{Variation der Panelanzahl}
\subsubsection{Rotation des Zylinders}
\section{Untersuchungen an ausgewählten Profilen}
\subsection{NACA0012-Profil}
\subsection{Joukowsky 12\%-Profil}
\subsection{Fehlerabschätzung des Hess-Smith-Verfahrens}




\newpage