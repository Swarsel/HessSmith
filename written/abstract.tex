\chapter*{\abstractname} % Korrekter Name für den Abstract in der jeweiligen Sprache
Im Zuge der vorliegenden Arbeit wurde ein Panelverfahren nach Hess-Smith zur Berechnung von ebenen Potentialströmungen um geschlossene Körper implementiert und auf verschiedene Probleme angewandt. Dazu wurde die Programmiersprache Python verwendet.
\\
Zunächst wurde eine Klasse für Panele und ihre Eigenschaften entworfen. Im Zuge dessen wurde eine Routine entwickelt, die jene Parameter in eine .csv-Datei ausgibt. Danach wurde eine Klasse für Tragflächenprofile sowie Methoden zur Berechnung der ihnen zugehörigen Systemparameter implementiert. Mit den somit berechneten Systemparametern kann das sich ergebende lineare Gleichungssystem für die Quellstärken gelöst werden; im Anschluss wurden aus den ermittelten Quellstärken die Tangentialgeschwindigkeiten und Druckbeiwerte an den jeweiligen Panelen, sowie unter Hinzunahme einer Wirbelbewegung der Auftriebsbeiwert für das Gesamtsystem ermittelt. Abschließend wurde eine Routine für die Ermittlung der Tangentialgeschwindigkeiten an beliebigen Punkten im Strömungsfeld entwickelt.
\\
Die fertige Implementation wurde verwendet, um für die Panele verschiedener Tragflächenprofile Graphen zu Geometrien und Parametern zu gewinnen. Anschließend wurde anhand eines Kreiszylinders mit 8 Panelen gleicher Seitenlänge der Vektor der Quellbelegung, sowie die Tangentialgeschwindigkeiten und Druckbeiwerte an den Panelmittelpunkten ermittelt sowie die Ergebnisse für den Druckbeiwert mit dem theoretischen Ergebnis aus der Potentialtheorie verglichen. Dabei wurde für die mittlere Abweichung in Abhängigkeit der Panelanzahl $n$ eine Abhängigkeit $\propto 1 / n^2$ ermittelt. Abschließend wurde das oben genannte System um eine Wirbelbelegung erweitert und der Auftriebsbeiwert unter verschiedenen Anstellwinkeln des Profils bei konstantem Angriffswinkel berechnet. 
\\
Letztlich wurde die Implementierung auf Joukowski-Profile angewandt und eine experimentelle Abschätzung der Fehlerordnung des Hess-Smith-Verfahrens durchgeführt. Dabei ergab sich eine relative Abweichung des Auftriebsbeiwerts des Hess-Smith Verfahrens vom theoretischen Resultat von $3.3 \%$.

\newpage
