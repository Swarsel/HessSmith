\appendix

\addsec{Anhang A: Lednicer- und Selig-Format}
\label{appendix:a}

Es folgt eine Gegenüberstellung des Selig-Formats (links) mit dem Lednicer-Format (rechts) am Beispiel des NACA M13-Profils.

\begin{minipage}{0.45\textwidth}
\verbatiminput{dataformats/m13_selig.DAT}
\end{minipage}
    \hfill
\begin{minipage}{0.45\textwidth}
\verbatiminput{dataformats/m13_lednicer.DAT}
\end{minipage}

\newpage
\addsec{Anhang B: Beispielausgabe der Methode .write\_panels()}
\label{appendix:b}
Hier ist die Ausgabe der ersten 30 Zeilen der AirfoilProfile-Methode .write\_panels() gezeigt, am Beispiel des NASA: HSNLF(1)-0213 Profils.

\verbatiminput{writtenvals/hsnlf213_vals.csv}

\newpage
\addsec{Anhang C: Codeausschnitte}
\label{appendix:c}
Hier sind ausgewählte Codeausschnitte gezeigt, welche zur Lösung der Problemstellungen geschrieben wurden.
\subsubsection{Die Klasse Panel}
Die Klasse Panel modelliert die Panels $\mathcal{C}_i$ eines gegebenen Profils. Gespeichert werden neben den charakteristischen Parametern $X_i, Y_i, \theta _i, l_i$ auch die Quellbelegung $q_i$, die Tangentialgeschwindigkeit $v_i^{(t)}$ unter gegebenen Anströmwinkel $\alpha $ und -geschwindigkeit $V_{\infty}$, und der resultierende Druckbeiwert $c_{p_i}$.
\begin{lstlisting}[language=Python]
class Panel:

    def __init__(self, xa, ya, xb, yb):
        self.xa, self.ya = xa, ya
        self.xb, self.yb = xb, yb

        self.xm = (xa + xb) / 2
        self.ym = (ya + yb) / 2
        self.length = np.sqrt((xb - xa) ** 2 + (yb - ya) ** 2)
        self.theta = np.arctan2(yb - ya, xb - xa)

        self.q = None
        self.vt = None
        self.cp = None
\end{lstlisting}

\subsubsection{Die Klasse AirfoilProfile}
Die Klasse AirfoilProfile modelliert ein gegebenes Profil. Sie speichert neben wählbaren Namen und der ihr zugewiesenen Panele auch die Profiltiefe $t$ und sämtliche Systemparameter $\xi_{ij}, \; \eta_{ij}, \; I_{ij}, \; J_{ij}, \; A_{ij}^{(n)}, \;A_{ij}^{(t)}, \;M_{ij}$. \\
Durch einen Aufruf der Methode .solve(V, a) mit gegebenen $\alpha $ und $V_{\infty}$ werden für alle zugeordneten Panele die Quellstärken, Tangentialgeschwindigkeiten und Druckbeiwerte berechnet (und in den jeweiligen Klassenvariablen abgespeichert). Dabei wird ebenfalls die Genauigkeit der Approximation $\sum q_i l_i$ berechnet. Die Methode kann mit dem Schlüsselwortargument vortex=True aufgerufen werden, wodurch ebenfalls der Auftriebsbeiwert $c_a$, sowie die Wirbelbewegung $\gamma$ \\ berechnet werden.
Die Methode .write\_panels() erzeugt eine .csv-Datei, welche für jedes Panel die Werte $X_i, Y_i \theta_i, l_i$ in eine Zeile schreibt (siehe \nameref{appendix:a}). \\
Die Methode .compute\_free\_vt(x, y, V, a) ermöglicht die Berechnung der Tangentialgeschwindigkeiten an jedem Punkt $(x_i, y_i)$ des Profils. Diese werden als Tupel von der Methode zurückgegeben.
\begin{lstlisting}[language=Python]
class AirfoilProfile:
    def __init__(self, panels, name=None, vortex=True):
        self.panels = panels
        self.name = name
        self.x = [panel.xa for panel in self.panels]
        self.y = [panel.ya for panel in self.panels]
        self.len = len(self.panels)
        self.vortex = vortex
        self.shape = Polygon([(a,b) for a,b in zip(self.x, self.y)])

        self.U = sum([panel.length for panel in self.panels])
        self.t = abs(max(self.x) - min(self.x))

        self.xi = self.__xi()
        self.eta = self.__eta()
        self.I = self.__i()
        self.J = self.__j()
        self.An = self.__an()
        self.At = self.__at()
        self.M = self.__m()

        self.solve_state = None
        self.ca = None
        self.accuracy = None
        self.gamma = None

    def write_panels(self, filename, n=2):
        header = ["X_i", "Y_i", "theta_i", "l_i"]
        with open(filename, "w+", encoding='UTF8',newline="") as file:
            writer = csv.writer(file)
            writer.writerow(header)
            for panel in self.panels:
                writer.writerow([round(panel.xm,n), round(panel.ym,n), round(panel.theta*180/np.pi,n), round(panel.length,n)])

    def __xi(self, x=None, y=None):
        panels = self.panels
        n_panels = len(panels)
        if x is not None and y is not None:
            n = 1
        else:
            n = len(panels)
        Xi = np.empty((n, n_panels), dtype=float)
        for i in range(n):
            for j in range(n_panels):
                if x is not None and y is not None:
                    i_xm, i_ym = x, y
                else:
                    i_xm, i_ym = panels[i].xm, panels[i].ym
                pj = panels[j]
                Xi[i][j] = (i_xm - pj.xm) * np.cos(pj.theta) + (i_ym - pj.ym) * np.sin(pj.theta)
        return Xi

    def __eta(self, x=None, y=None):
        panels = self.panels
        n_panels = len(panels)
        if x is not None and y is not None:
            n = 1
        else:
            n = len(panels)
        Eta = np.empty((n, n_panels), dtype=float)
        for i in range(n):
            for j in range(n_panels):
                if x is not None and y is not None:
                    i_xm, i_ym = x, y
                else:
                    i_xm, i_ym = panels[i].xm, panels[i].ym
                pj = panels[j]
                Eta[i][j] = - (i_xm - pj.xm) * np.sin(pj.theta) + (i_ym - pj.ym) * np.cos(pj.theta)
        return Eta

    def __i(self, Xi=None, Eta=None):
        panels = self.panels
        if Xi is None and Eta is None:
            Xi = self.xi
            Eta = self.eta
        n_panels = len(panels)
        n = Xi.shape[0]
        II = np.empty((n, n_panels), dtype=float)
        for i in range(n):
            for j in range(n_panels):
                pj = panels[j]
                if i == j and n > 1:
                    II[i][j] = 0
                else:
                    II[i][j] = (1 / (4 * np.pi)) * np.log(((pj.length + 2 * Xi[i][j]) ** 2 + 4 * (Eta[i][j] ** 2)) / ((pj.length - 2 * Xi[i][j]) ** 2 + 4 * (Eta[i][j] ** 2)))
        return II

    def __j(self, Xi=None, Eta=None):
        panels = self.panels
        if Xi is None and Eta is None:
            Xi = self.xi
            Eta = self.eta
        n_panels = len(panels)
        n = Xi.shape[0]
        JJ = np.empty((n, n_panels), dtype=float)
        for i in range(n):
            for j in range(n_panels):
                pj = panels[j]
                if i == j and n > 1:
                    JJ[i][j] = 0.5
                else:
                    JJ[i][j] = (1 / (2 * np.pi)) * np.arctan((pj.length - 2 * Xi[i][j])/ (2 * Eta[i][j])) + (1 / (2 * np.pi)) * np.arctan((pj.length + 2 * Xi[i][j])/ (2 * Eta[i][j]))
        return JJ

    def __an(self, II=None, JJ=None):
        panels = self.panels
        if II is None and JJ is None:
            II = self.I
            JJ = self.J
        n_panels = len(panels)
        n = II.shape[0]
        if n == 1:
            i_theta = 0
        AN = np.empty((n, n_panels), dtype=float)
        for i in range(n):
            for j in range(n_panels):
                if n > 1:
                    i_theta = panels[i].theta
                pj = panels[j]
                AN[i][j] = - np.sin(i_theta - pj.theta) * II[i][j] + np.cos(i_theta - pj.theta) * JJ[i][j]
        return AN

    def __at(self, II=None, JJ=None):
        panels = self.panels
        if II is None and JJ is None:
            II = self.I
            JJ = self.J
        n_panels = len(panels)
        n = II.shape[0]
        if n == 1:
            i_theta = 0
        AT = np.empty((n, n_panels), dtype=float)
        for i in range(n):
            for j in range(n_panels):
                if n > 1:
                    i_theta = panels[i].theta
                pj = panels[j]
                AT[i][j] = np.cos(i_theta - pj.theta) * II[i][j] + np.sin(i_theta - pj.theta) * JJ[i][j]
        return AT

    def __m(self):
        panels = self.panels
        AN = self.An
        AT = self.At
        vortex = self.vortex
        n = len(panels)
        if vortex:
            MM = np.empty((n + 1, n + 1), dtype=float)
        else:
            MM = np.empty((n, n), dtype=float)
        if vortex:
            MM[:-1, :-1] = AN
            MM[:-1, -1] = np.sum(AT, axis=1)

            r = np.empty(n + 1, dtype=float)
            r[:-1] = AT[0, :] + AT[n - 1, :]
            r[-1] = -np.sum(AN[0, :] + AN[n - 1, :])
            MM[-1, :] = r
        else:
            MM = AN
        return MM

    def solve(self, V=1, a=5):
        a = np.radians(a)
        b = self.__b(V, a)

        self.__q(b)  # setzt auch self.gamma
        self.__vt(V, a)
        self.__cp(V)
        self.__ca(V)
        self.__accuracy()
        self.solve_state = (V,a)

    def compute_free_vt(self, x, y, V=1, a=5):
        if (V,a) != self.solve_state:
            self.solve(V=V, a=a)
        a = np.radians(a)
        point = Point(x,y)
        if point.within(self.shape) or self.shape.touches(point):
            vtx = np.nan
            vty = np.nan
        else:
            xi = self.__xi(x=x, y=y)
            eta = self.__eta(x=x, y=y)
            I = self.__i(Xi=xi, Eta=eta)
            J = self.__j(Xi=xi, Eta=eta)
            An = self.__an(II=I, JJ=J)
            At = self.__at(II=I, JJ=J)

            vtx = sum([At[0][j] * self.panels[j].q for j in range(self.len)]) - self.gamma * sum(
                [An[0][j] for j in range(self.len)]) + V * np.cos(a)

            vty= sum([An[0][j] * self.panels[j].q for j in range(self.len)]) + self.gamma * sum(
                [At[0][j] for j in range(self.len)]) + V * np.sin(a)

        return vtx, vty

    def __b(self, V, a):
        panels = self.panels
        vortex = self.vortex
        n = len(panels)
        if vortex:
            B = np.empty(n + 1, dtype=float)
        else:
            B = np.empty(n, dtype=float)
        for i in range(n):
            pi = panels[i]
            B[i] = -V * np.sin(a - pi.theta)
        if vortex:
            B[-1] = -V * (np.cos(a - panels[0].theta) + np.cos(a - panels[-1].theta))
        return B

    def __q(self, B):
        panels = self.panels
        qs = np.linalg.solve(self.M, B)

        for i, panel in enumerate(panels):
            panel.q = qs[i]
        if self.vortex:
            self.gamma = qs[-1]

    def __vt(self, V, a):
        panels = self.panels
        AN = self.An
        AT = self.At
        n = len(panels)
        if self.vortex:
            VT = np.empty(n, dtype=float)
            for i in range(n):
                VT[i] = sum([AT[i][j] * panels[j].q for j in range(n)]) \
                        - self.gamma * sum([AN[i][j] for j in range(n)]) \
                        + V * np.cos(a - panels[i].theta)
        else:
            VT = np.empty(n, dtype=float)
            for i in range(n):
                VT[i] = sum([AT[i][j] * panels[j].q for j in range(n)]) \
                        + V * np.cos(a - panels[i].theta)

        for i, panel in enumerate(panels):
            panel.vt = VT[i]

    def __cp(self, V):
        for panel in self.panels:
            panel.cp = 1 - (panel.vt / V) ** 2

    def __ca(self, V):
        self.ca = 2 / (V * self.t) * sum([panel.vt * panel.length for panel in self.panels])

    def __accuracy(self):
        self.accuracy = sum([panel.q * panel.length for panel in self.panels])


\end{lstlisting}

\subsubsection{make\_cylinder(r,n)}
Diese Funktion wurde verwendet, um die $x$- und $y$-Koordinaten eines Kreiszylinder mit Radius $r$ und $n$ gleich langen Panels zu generieren. Es wurde dabei besonderes Augenmerk darauf gelegt, dass es an den Endpunkten durch Rundungsfehler nicht zu einem disjunkten Körper kommt.
\begin{lstlisting}
def cylinder(r=1, n=8):
    a = np.linspace(0, 360, num=n+1, endpoint=True) / 180 * np.pi

    x = r * np.cos(a)
    y = r * np.sin(a)
    if abs(x[0] - x[-1]) <= 10 ** (-15):
        x[-1] = x[0]
    if abs(y[0] - y[-1]) <= 10 ** (-15):
        y[-1] = y[0]

    return x, y
\end{lstlisting}

\subsubsection{make\_panels(x,y, reverse=False)}
\begin{lstlisting}
    if type(x) is not np.ndarray:
        x = np.array(x)
    if type(y) is not np.ndarray:
        y = np.array(y)
    if (x[0], y[0]) != (x[-1], y[-1]):
        x = np.append(x, x[0])
        y = np.append(y, y[0])
    if reverse:
        x = np.flipud(x)
        y = np.flipud(y)
    n = len(x) - 1
    panels = np.array([Panel(x[i], y[i], x[i + 1], y[i + 1]) for i in range(n)])
    return panels
\end{lstlisting}

\subsubsection{parsecoords(filename)}
\begin{lstlisting}
    x, y = np.loadtxt(filename, dtype=float, unpack=True)
    return x, y
\end{lstlisting}

\subsubsection{Joukowsky-Profil-Generator}

\begin{lstlisting}
def joukowski_transfrom(zeta):
    z = zeta + 1 / zeta
    return z
def make_joukowski(mux=0.2, muy=0.1, N=100):
    center = -mux + muy * 1j 
    R = np.sqrt((1 + mux) ** 2 + muy ** 2)

    theta = np.linspace(0, 2 * np.pi, N)
    Xc = np.real(center) + R * np.cos(theta)
    Yc = np.imag(center) + R * np.sin(theta)

    p = joukowski_transfrom(Xc + Yc * 1j)
    Xp, Yp = np.real(p), np.imag(p)
    return Xp, Yp, R, muy
\end{lstlisting}

\subsubsection{Kármán-Trefftz-Profil-Generator}

\begin{lstlisting}
def karman_trefftz_transform(zeta, n):
    z = n * ((zeta + 1) ** n + (zeta - 1) ** n) / ((zeta + 1) ** n - (zeta - 1) ** n)
    return z
def make_karman_trefftz(mux=0.2, muy=0.1, n=1.9, N=100):
    center = -mux + muy * 1j
    R = np.sqrt((1 + mux) ** 2 + muy ** 2) 

    theta = np.linspace(0, 2 * np.pi, N)
    Xc = np.real(center) + R * np.cos(theta)
    Yc = np.imag(center) + R * np.sin(theta)
    tx = np.real(center)
    ty = np.imag(center)

    p = karman_trefftz_transform(Xc + Yc * 1j, n)
    Xp, Yp = np.real(p), np.imag(p)
    return Xp, Yp, R, muy
\end{lstlisting}


