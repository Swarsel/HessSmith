\chapter{Aufgabenstellung}
Die Berechnung der aerodynamischen Eigenschaften eines Tragflächenprofils.

\section{Zielsetzung}
Es sollen folgende Punkte implementiert werden:
\begin{itemize}
  \item Eine Klasse für Panele und die sie definierenden Eigenschaften: Mittelpunkte auf der $x$- und $y$-Achse, Neigungswinkel $\theta_{i}$ und Länge $l_i$. Jedes Panel enthält ebenfalls Möglichkeiten zur Speicherung der zugehörigen Quellstärken $q_i$, Tangentialgeschwindigkeiten $v_i^{(t)}$ und Druckbeiwerten $c_{p_i}$
  \item Eine Klasse für Trägerprofile, ihre Umfänge $U$ und Tiefe $t$ sowie mathematisch relevante Systemparameter $\xi_{ij}, \; \eta_{ij}, \; I_{ij}, \; J_{ij}, \; A_{ij}^{(n)}, \;A_{ij}^{(t)}, \;M_{ij}$.
  \item Eine Methode zur Lösung des linearen Gleichungssystems $M \vec q = \vec b$ für den Vektor der Quellbewegung unter einer konstanten Anströmgeschwindigkeit $V_{\infty}$ und einem Anstellwinkel des Profils $\alpha $, sowie der Berechnung der daraus resultierenden Tangentialgeschwindigkeiten und Druckbeiwerte.
\end{itemize}
Weiters soll mittels der Implementation eine Untersuchung der Abweichung der ermittelten Druckbeiwerte von den aus der Potentialtheorie berechneten Werten für einen achtseitigen Kreiszylinder erfolgen.

\section{Methodik}
Zur Lösung der Aufgabenstellung wurde die Programmiersprache Python verwendet. Für die Lösung der linearen Gleichungssysteme wird linalg.solve\footnote{linalg.solve verwendet die LAPACK routine \_gesv als Solver \cite{harris2020array}} aus der Python-Bibliothek NumPy verwendet.\\Sämtliche Graphen wurden mithilfe der Python-Bibliothek Matplotlib erstellt. \cite{Hunter:2007} Sofern analytische Lösungen zu den linearen Gleichungssystemen berechnet wurden, wurde dafür die Python-Bibliothek SciPy verwendet. \cite{2020SciPy-NMeth}