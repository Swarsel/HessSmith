\chapter{Aufgabenstellung}
%Die Berechnung der aerodynamischen Eigenschaften eines Tragflächenprofils.
%
\section{Zielsetzung}
Das Ziel der vorliegenden Arbeit ist die Berechnung  aerodynamischer Eigenschaften von Tragflächenprofilen. \\ Dazu soll eine Klasse implementiert werden, welche die Tragflächenprofile abbildet. Mithilfe dieser Klasse sollen unter gegebenen Parametern Berechnungen zu den Druckbeiwerten und Tangentialgeschwindigkeit auf der Oberfläche sowie dem Auftriebsbeiwert angestellt werden. 
Weiters soll eine Untersuchung der Abweichung der ermittelten Druckbeiwerte von den aus der Potentialtheorie berechneten Werten für einen achtseitigen Kreiszylinder sowie eine Ermittlung der relevanten Größen unter verschiedenen Anstellwinkeln des Kreiszylinders erfolgen. \\
Für allgemeine Tragflächenprofile sollen Funktionen zur Berechnung und Darstellung des Umströmungsprofils sowie der Druckbeiwerte implementiert werden und diese auf ausgewählte Profile angewandt werden. \\
Abschließend soll die Abweichung des mittels des Panelverfahrens berechneten Auftriebsbeiwerts eines Joukowski-Profils vom analytischen Resultat untersucht werden.

\section{Methodik}
Zur Lösung der Aufgabenstellung wurde die Programmiersprache Python (Version 3.8) verwendet, \cite{python2009}. \\Für die Lösung der linearen Gleichungssysteme wurde linalg.solve\footnote{linalg.solve verwendet die LAPACK routine \_gesv als Solver,  \cite{harris2020array}} aus der Python-Bibliothek NumPy (Version 1.23.1) verwendet, \cite{harris2020array}. Sämtliche Graphen wurden mithilfe der Python-Bibliothek Matplotlib (Version 3.5.2) erstellt, \cite{Hunter:2007}. Analytische Lösungen zu den linearen Gleichungssystemen  wurden mit der Python-Bibliothek SymPy (Version 1.10.1) berechnet, \cite{Sympy}. Für die Kurvenanpassung in \ref{chap:analyticalcylinder} und \ref{chap:profilauswertung} wurde die Python-Bibliothek SciPy (Version 1.8.1) verwendet, \cite{2020SciPy-NMeth}. Die Maskierung der Trägerprofile für die Strömungs- und Konturgraphen in \ref{chap:profilauswertung} wurde mithilfe der Python-Bibliothek shapely (Version 1.8.2) erreicht, \cite{shapely2007}.