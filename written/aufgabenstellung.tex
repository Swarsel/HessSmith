\chapter{Aufgabenstellung}
%Die Berechnung der aerodynamischen Eigenschaften eines Tragflächenprofils.
%
\section{Zielsetzung}
Das Ziel der vorliegenden Arbeit ist die Berechnung  aerodynamischer Eigenschaften von Tragflächenprofilen. Dazu sollen folgende Punkte implementiert werden:
\begin{itemize}
  \item Eine Klasse für Panele und die sie definierenden Eigenschaften: Mittelpunkte auf der $x$- und $y$-Achse, Neigungswinkel $\theta_{i}$ und Länge $l_i$. %Jedes Panel enthält ebenfalls Möglichkeiten zur Speicherung der zugehörigen Quellstärken $q_i$, Tangentialgeschwindigkeiten $v_i^{(t)}$ und Druckbeiwerten $c_{p_i}$
  \item Eine Klasse für Trägerprofile, ihre Umfänge $U$ und Tiefe $t$ sowie mathematisch relevante Systemparameter $\xi_{i,j}, \; \eta_{i,j}, \; I_{i,j}, \; J_{i,j}, \; A_{i,j}^{(n)}, \;A_{i,j}^{(t)}, \;M_{i,j}$.
  \item Eine Methode zur Lösung des linearen Gleichungssystems $M \vec q = \vec b$ für den Vektor der Quellbelegung unter einer konstanten Anströmgeschwindigkeit $V_{\infty}$ und einem Anstellwinkel des Profils $\alpha $, sowie der Berechnung der daraus resultierenden Tangentialgeschwindigkeiten und Druckbeiwerte.
\end{itemize}
Weiters soll eine Untersuchung der Abweichung der ermittelten Druckbeiwerte von den aus der Potentialtheorie berechneten Werten für einen achtseitigen Kreiszylinder sowie die Abweichung des Auftriebsbeiwerts eines Joukowski-Profils vom analytischen Resultat erfolgen.

\section{Methodik}
Zur Lösung der Aufgabenstellung wurde die Programmiersprache Python (Version 3.8) verwendet. \cite{python2009} \\Für die Lösung der linearen Gleichungssysteme wurde linalg.solve\footnote{linalg.solve verwendet die LAPACK routine \_gesv als Solver \cite{harris2020array}} aus der Python-Bibliothek NumPy (Version 1.23.1) verwendet. \cite{harris2020array} Sämtliche Graphen wurden mithilfe der Python-Bibliothek Matplotlib (Version 3.5.2) erstellt. \cite{Hunter:2007} Analytische Lösungen zu den linearen Gleichungssystemen  wurden mit der Python-Bibliothek SymPy (Version 1.10.1) berechnet. \cite{Sympy} Für die Kurvenanpassung in \ref{chap:analyticalcylinder} wurde die Python-Bibliothek SciPy (Version 1.8.1) verwendet. \cite{2020SciPy-NMeth} Die Maskierung der Trägerprofile für die Strömungs- und Konturgraphen in \ref{chap:profilauswertung} wurde mithilfe der Python-Bibliothek shapely (Version 1.8.2) erreicht. \cite{shapely2007}