\usepackage{ifpdf}
\ifpdf
  \input{glyphtounicode.tex}    %Part of modern distribution
  \input{glyphtounicode-cmr.tex}     %Additionnal glyph: You must grab it from pdfx package
  \pdfgentounicode=1
  \pdfinterwordspaceon
  \usepackage[a-2u,pdf17]{pdfx}
  \pdfomitcharset 1
\else  %Place here the settings for other compilator
\fi
%Encoding + cmap (to get proper UTF8 mapping)
%------------------------------------------------------
\usepackage{cmap}
\usepackage[utf8]{inputenc} % Richtiges anzeigen von Umlauten und quasi allen anderen Schriftzeichen
\usepackage[T1]{fontenc} % Wichtig für alles was mehr als ASCII verwendet
\usepackage{csquotes} % Schöne Anführungsstriche mit \enquote{Text}
\usepackage{amsmath} % Bessere und schönere mathematische Formeln
\usepackage{mathtools} % Noch schönerere mathematische Formeln
\usepackage{amstext} % \text{} Macro in mathematischen Formeln
\usepackage{amsfonts} % Erweiterte Zeichensätze für mathematische Formeln
\usepackage{amssymb} % Spezielle mathematische Symbole.
\usepackage{booktabs}
%Correct UTF8 mapping for ams fonts
\ifdefined\pdffontattr% \ifdefined is part of the e-TeX extension, which is part of any modern LaTeX compiler.
    \immediate\pdfobj stream file {umsa.cmap}
    {\usefont{U}{msa}{m}{n}\pdffontattr\font{/ToUnicode \the\pdflastobj\space 0 R}}
    \immediate\pdfobj stream file {umsb.cmap}
    {\usefont{U}{msb}{m}{n}\pdffontattr\font{/ToUnicode \the\pdflastobj\space 0 R}}
\fi
\usepackage{array} % Matrizen in mathematischen Formeln
\usepackage{textcomp} % Für textmu und textohm etc. um im Fließtext keine Mathematik 
\usepackage{textalpha} % Damit können griechische Zeichen direkt im Text verwendet werden (siehe zeichen.txt)
\usepackage{paralist} % Für compactitem und compactenum
\usepackage{xstring} % Für IF in Titelseite

\usepackage[version=3]{mhchem} % Für Chemische Formeln
\usepackage{braket} % Für das quantenmechanische Bra-Ket

\usepackage{geometry} % Seitenränder und Seiteneigenschaften setzen
%\usepackage[showframe]{geometry} % Anzeigen der Seitenränder, nützlich für debugging. http://ctan.org/pkg/geometry

\usepackage[bottom]{footmisc} % Zwingt Fußnoten an das Ende der Seite
\usepackage[pdftex]{hyperref} % Links richtig anzeigen. Sowohl innerhalb des Dokuments (Fußzeilen, Formeln), als auch ins Internet

\usepackage[ % Biblatex für die Zitate und Referenzen
	backend=biber,
	hyperref=true
		]{biblatex}

\usepackage{xkeyval} % Erlaubt "Variablen" zu definieren, wird für Titelseite gebraucht
\usepackage{graphicx} % Wichtig für das Einbinden von Grafiken
\usepackage{caption}
\usepackage{subcaption} % Einbinden von mehreren Grafiken in einer figure

\usepackage{dirtree} % Erlaubt das erstellen von Dateibäumen
% \dirtreecomment{Text} erstellt einen Kommentar zu dem Verzeichnis bzw. der Datei
\newcommand{\dirtreecomment}[1]{\dotfill{} \begin{minipage}[t]{0.5\textwidth}#1\end{minipage}}

\usepackage{fancyvrb} % Mehr Optionen für Verbatim
\usepackage{listings} % Zur Darstellung von Programmcode
\usepackage{pdflscape} % Querformat Seiten
\usepackage{verbatim} % txt input

\lstset{language=Python,
  breaklines=true,
  showstringspaces=false,
  columns=flexible,
  numbers=none,
  tabsize=2
}

\newcommand{\writeIn}[1]{\usepackage[#1]{babel}} % Definiert einen neuen Befehl um die Sprache des Dokuments zu setzen
\DefineBibliographyStrings{german}{ 
	andothers = {{et\,al\adddot}},             
}
\DeclareLabelalphaTemplate{ 
\labelelement{
\field[final]{shorthand} 
\field{label} 
\field[strside=left,ifnames=1]{labelname} 
}    
\labelelement{
\field[strside=right]{year}
}}
\renewcommand*{\labelalphaothers}{~et al.~}
\usepackage{float}
\usepackage{colorprofiles}
\PassOptionsToPackage{usenames,dvipsnames}{color}
\usepackage{color} % Farben für den todo Befehl
\newcommand{\todo}[1]{{\color{Cerulean}(TODO: #1)}} % Einfach \todo{Text} verwenden!

\newcommand{\blankpage}{ \newpage \thispagestyle{empty} \mbox{} \newpage }
